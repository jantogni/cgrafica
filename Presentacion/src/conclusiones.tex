\frame{
\frametitle{Conclusiones}

Las conclusiones del trabajo específicas fueron:

\begin{itemize}
	\item Blending: habilitar el blend dentro de OpenGL permite bastantes funcionalidades que de manera
		visual son útiles como en el caso de transparencia. Para este caso es importante entender la
		distribución en el buffer de color de un pixel y que este tiene 4 capas (color + opacidad).
	\item Antialiasing: este fenómeno aparece en distintas áreas de la ingeniería, pero del punto de vista de computación
		gráfica, es bueno saber que con un par de líneas OpenGL permite realizar cálculos internos para evitar el problema.
		Además las diferencias entre como aplicar antialiasing entre puntos, líneas, polígonos es bastante similar.
\end{itemize}

}
